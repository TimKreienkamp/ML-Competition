\documentclass[paper=a4, fontsize=11pt]{scrartcl} 
\usepackage[T1]{fontenc} 
\usepackage{fourier} 
\usepackage[english]{babel} 
\usepackage{amsmath,amsfonts,amsthm} 
\usepackage{listings}

\usepackage{sectsty} % Allows customizing section commands
\allsectionsfont{\centering \normalfont\scshape} % Make all sections centered, the default font and small caps

\usepackage{fancyhdr} % Custom headers and footers
\pagestyle{fancyplain} % Makes all pages in the document conform to the custom headers and footers
\fancyhead{} % No page header - if you want one, create it in the same way as the footers below
\fancyfoot[L]{} % Empty left footer
\fancyfoot[C]{} % Empty center footer
\fancyfoot[R]{\thepage} % Page numbering for right footer
\renewcommand{\headrulewidth}{0pt} % Remove header underlines
\renewcommand{\footrulewidth}{0pt} % Remove footer underlines
\setlength{\headheight}{13.6pt} % Customize the height of the header

\numberwithin{equation}{section} % Number equations within sections (i.e. 1.1, 1.2, 2.1, 2.2 instead of 1, 2, 3, 4)
\numberwithin{figure}{section} % Number figures within sections (i.e. 1.1, 1.2, 2.1, 2.2 instead of 1, 2, 3, 4)
\numberwithin{table}{section} % Number tables within sections (i.e. 1.1, 1.2, 2.1, 2.2 instead of 1, 2, 3, 4)

\setlength\parindent{0pt} % Removes all indentation from paragraphs - comment this line for an assignment with lots of text


\newcommand{\horrule}[1]{\rule{\linewidth}{#1}} % Create horizontal rule command with 1 argument of height

\title{	
\normalfont \normalsize 
\textsc{Barcelona Graduate School of Economics - Data Science} \\ [25pt]
\horrule{2pt} \\[0.5cm] 
\huge Deep Feature Learning for (Random) Forest Cover Type Prediction  \\ 
\horrule{2pt} \\[0.5cm] 
}

\author{Jessica Leal Tim Kreienkamp Philipp Schmidt} 
\date{}

\begin{document}

\maketitle
\section{Introduction}
\subsection{Dataset Characteristics}
Here we can explain how the data is, the sources, etc. An introduction of the data \textbf{max five lines!}
\subsection{Establishing a Baseline}
Initially, we need to establish a benchmark. This is necessary for us to judge all future modeling attempts. A very first, absolute worst case benchmark has already been established by the instructors - The random submission gives an accuracy of about 0.14. We run a random forest in R with no additional parameter tuning on the full dataset. This gives us an accuracy of 0.80. From here on, we intend to make improvements.

\section{Exploratory Analysis}
\section{Experiments}
In a first set of experiments, we consider several well established classes of hypothesis functions. In particular, we consider Random Forests, Support Vector Machines, k-NN,  Gradient Tree Boosting and Neural Nets (Deep Learning). 
We first run these over a grid of hyperparameters without any feature engineering. This gives us a feeling for how well different methods are suited for the task at hand. In grid search, one evaluates a classifier over a grid, i.e. the cartesian product of several hyper parameter ranges. This process is computationally intensive, but since the data set is small, we can afford this luxury. The code for these experiments can be found in the files \lstinline{h2o_gridsearches.R}, \lstinline{svm_experiments.R} and \lstinline{knn_experiments.R}. The following figure shows the accuracy of the best model in each class. All accuracy values are obtained by a 10-fold cross validation procedure. 

We clearly see that the performance of SVM and k-NN, at least in the tried configuration with this set of features is rather disappointing. Tree based methods  and neural nets however, show - with some parameter tuning - a significant improvement over the baseline. 

\section{Feature Engineering and Model Tuning}
We settle on Deep Learning and Random Forests, since those performed best in the initial series of experiments. Based on our exploratory analysis, we add three new features to the data set: Horizontal Distance to Hydrology (water), Vertical Distance to Hydrology, and Total Distance to Hydrology. We feed these into the our previously best performing models, which leaves us with a slight gain in accuracy. 
Since Deep Learning seems to perform well for this data set, we try a heuristic: Why not train the Random Forest on features extracted with Deep Learning? The recent success on Deep Learning seems to be partly based on it's ability to abstract high-level concepts from the data. For this reason, Machine Learning researchers use it for feature, or representation learning. Feature learning can be done in a supervised or unsupervised manner. We use it in a supervised manner based on the heuristic: the neural net with the highest accuracy will provide the best features for our random forest. Operatively, what is done is just that the nodes of the last hidden layer in the neural net are extracted and used to transform the original feature space. The following figure shows the accuracy of all methods described: Random Forests, Deep Learning, and Random Forests with "Deep Features". 


We see that this results in a significant gain in accuracy. Apparently, Deep Learning is able to extract useful features from the data. However, in our last example we trained the random forest on 200 features. Too many features may potentially cause overfitting. This is why, in the next series of experiments, we add another layer to the neural net, where the size is significantly reduced, thus enforcing sparseness in the output. We try this with 50, 30 and 20 neurons. 


Inspecting the output shows that


\end{document}